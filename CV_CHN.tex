%!TEX TS-program = xelatex
%!TEX encoding = UTF-8 Unicode
% Use Xelatex and UTF-8

\documentclass[CHN]{Sketch}
% 用这一句引用SealZhang类 选项不填默认英文 可以写[ENG]或者[CHN]

\titleformat{\section}{\Large\scshape\raggedright}{}{0em}{}[\titlerule]
\titlespacing{\section}{0pt}{3pt}{3pt}

\begin{document}

%-----------------TITLE-------------
\par{\centering
		{%\includegraphics[width = 0.2\textwidth]{photo.jpg} \quad
		\Huge \textsc{张楚珩} }
		\bigskip\par}



%--------------SECTIONS---------
\section{基本信息}

\begin{tabular}{rl}
    \textsc{出生地与出生日期} & 中国湖北省  | xxxx年xx月xx日 \\
    \textsc{通信地址}   & 中华人民共和国江苏省南京市 \\
                        & 鼓楼区汉口路22号 南京大学鼓楼校区\\
                        & 南苑四舍 430\\
    \textsc{手机}     & +86 158xxxxxxxx\\
    \textsc{邮箱}     & xxxxxxxx@xxx.com \\
    \textsc{个人主页:}  & \href{http://sealzhang.tk}{sealzhang.tk(欢迎访问个人主页以获取更多信息)} 
\end{tabular}

%Section: Education
\section{教育经历}
\begin{tabular}{rl}	
2012年9月 & 本科\\
至今 & 南京大学 物理学院 光电科学系\\
&\normalsize \textsc{\underline{绩点}} 4.564/5.0 91.28/100 \textsc{\underline{排名}} 5/152 (截止大三结束)\\
\\
2006年9月 & 初中及高中\\
至2012年9月 & 四川省成都市成都外国语学校\\

\end{tabular}

%Section: Work Experience at the top
\section{科研经历}
\begin{tabular}{p{2cm}|p{12cm}}

\textsc{2014年5月} 
& 菲涅尔非相干全息(FINCH)的高分辨率成像\\
&\emph{南京大学 物理学院 光电科学系}\\
& 指导老师:丁剑平教授 \\
&\footnotesize{此项研究的目的是提高菲涅尔非相干全息的成像分辨率。此项目分为实验和模拟两个部分,本人主要负责MATLAB程序模拟部分。在模拟中,主要通过三步相移等方法消除成像中的零级像和孪生像等,以提高成像分辨率。同时,程序上还基于标量光学理论对整个实验光路进行了模拟。与此相关的研究还包括大学物理实验课上的自主探究部分以及现代光学课后的自主探究等。}\\
\multicolumn{2}{c}{} \\
\end{tabular}

\begin{tabular}{p{2cm}|p{12cm}}

\textsc{2015年暑假} & 西澳大学暑期科研项目 \\
&\emph{西澳大学 电子工程学院 光学与生物医学工程实验室}\\
& 指导老师: David Sampson 教授 \\
&\footnotesize{此课题组目前研究基于OCT的针尖成像技术,用于医学等领域。我在此课题组期间,主要做基于机器学习的图形分割,用于识别与TEM中得到的生物组织图像,以便于可以组的进一步研究。在此期间我开发了基于MATLAB以及Weka的工具,并且在Drosophila Larve神经组织的ssTEM图像上试验并且取得良好的效果。}\\

\multicolumn{2}{c}{} \\
\end{tabular}

\begin{tabular}{p{2cm}|p{12cm}}

\textsc{2015年9月} & 毕业设计:演化算法在深层网络中的应用\\
&\emph{南京大学 计算机科学以技术系 机器学习与数据挖掘课题组}\\
& 指导老师:俞扬副教授 \\
&\footnotesize{传统BP算法在深层网络的学习过程中遇到了一些障碍,我们试图利用演化算法的一些思想来解决深层网络的学习问题。目前我正工作于此项目。}\\

\end{tabular}

\section{社团活动}
\begin{tabular}{p{2cm}|p{12cm}}

\textsc{2012年} & 物理学院学生会宣传部部员 和 南京大学支点学社技术部部员 \\
 
 \multicolumn{2}{c}{} \\
 
\textsc{2013年} &  物理学院支点学社副社长 和 大学生创新训练计划学社主席团成员 \\

 \multicolumn{2}{c}{} \\
 
\textsc{2014年3月} &  果壳网万有青年烩@南京大学之时空侦探 三位主要策划人之一   \\ &活动于2014年3月29日在南京先锋书店举行\\

 \multicolumn{2}{c}{} \\
 
\textsc{2014年5月-2015年10月} &  南京大学物理学院“物理百年,口述历史”活动 六位主要策划人之一 
合著出版物: 高等教育出版社《南京大学物理学科口述史》 \\

\end{tabular}

%Section: Scholarships and additional info
\section{荣誉与竞赛}
\begin{tabular}{rll}
    \textsc{2012年度} & 国家奖学金\footnotesize(8,000 \textsc{RMB})\normalsize & 前1\%\\
    \textsc{2013年度} & 兴业责任奖学金\footnotesize(2,500 \textsc{RMB})\normalsize & 前5\%\\
    \textsc{2013年度} & 人民奖学金社会工作特长奖 & 前15\%\\
    \textsc{2013年11月} & University Physics Competition(UPC) \footnotesize (USA) \normalsize & 银奖 \\
    \textsc{2014年9月} & 大学生数学建模竞赛 \footnotesize (CHN, 队长) \normalsize & 二等奖 \\
     \textsc{2015年2月} & Mathematical Contest in Modeling (MCM)    \footnotesize (USA,队长) \normalsize &  Meritorious\\
\end{tabular}

%Section: Languages
\section{专业技能}
\begin{tabular}{rl}
 \textsc{编程与软件:}&C/C++(熟练), C\#, PHP, MySQL, HTML, \\& \LaTeX (熟练), MATLAB(熟练),\\ & Wolfram Mathematica, COMSOL\\
\textsc{理论与实践:}&数学模型建立与模拟, \\&光学光路搭建\\&机器学习\\
\end{tabular}

\section{标准化考试}
\begin{tabular}{l|ll}
 \textsc{2013年11月} & TOEFL(iBT) & 92(R24, L26, S20, W22) \\
 \textsc{2014年12月} & GRE & V153, Q170, AW4.0 \\
 \textsc{2013年6月} & 公共英语四级考试 & 585 \\
 \textsc{2013年12月} & 公共英语六级考试 & 532 \\
 \textsc{2013年12月} & 江苏省计算机等级考试 & 优秀 (二级, C语言) \\
\end{tabular}

\clearpage


\end{document}
